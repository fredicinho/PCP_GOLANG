\documentclass[12pt,titlepage]{article}





%PACKAGES
\usepackage[ngerman]{babel}
\usepackage[utf8]{inputenc}
\usepackage[a4paper,lmargin={2.5cm},rmargin={2.5cm},
tmargin={2.5cm},bmargin = {2.5cm}]{geometry}
\usepackage{graphicx}
\usepackage{caption}
\usepackage{float}
\usepackage{hyperref}
\usepackage{graphicx,wrapfig,lipsum}
\usepackage{wasysym}
\parindent0pt



\begin{document}



\thispagestyle{empty}

%TITELSEITE
\begin{center}
\textbf{Hochschule Luzern}\\
Departement für Informatik\\[12\baselineskip]

\begin{Huge}
PCP Projekt - Programmiersprachanalyse
\end{Huge} \\[6\baselineskip]

\begin{large}
\textbf{Analyse der Programmiersprache GO}
\end{large} \\[6\baselineskip]

\begin{large}
\textbf{Studierende}: Frederico Fischer, Oliver Werlen\\
\textbf{Dozenten}: Prof. Dr. Ruedi Arnold, Marcel Baumann \\

\textbf{Abgabedatum}: 26. Mai 2021 \\
\end{large}
\end{center}
\newpage


\section{Einleitung}
In dieser Projektarbeit wurde die Programmiersprache \textbf{Go} analysiert.
Es handelt es sich, wie auch bei vielen anderen Sprachen, um eine Multiparadigmensprache (funktional, imperativ und objektorientiert).
Dabei wurde Bezug auf dessen Entstehungsgeschichte sowie Spezialitäten und Feinheiten genommen.
Motivierungsgründe für die Wahl dieser Sprache waren, dass es sich dabei um eine sehr junge und oft verwendete Sprache handelt, die bei vielen neuen Anwendungen genutzt wird und eine hohe Popularität aufweist.

\section{Vision, Geschichte \& Verbreitung}
Die Sprache \glqq Go\grqq{} ist durch Google von den Entwicklern Robert Griesemer, Rob Pike und Ken Thompson entstanden.
Gründe für dessen Entstehung waren die Beseitigung der Langsam- sowie Schwerfälligkeiten und die Erhöhung der Produktivität und Skalierbarkeit ihrer Prozesse.
Das Zielpublikum dieser Sprache sind insbesondere Software Entwickler, welche sich mit grossen Softwaresystemen auseinandersetzen.
Diese Sprache zeichnet sich durch seine Performance, Lesbarkeit, unkomplizierte Einbindung von Abhängigkeiten sowie simple Nebenläufigkeitsimplementierung aus.
Insbesondere in der DevOps-Gemeinde wurde diese sehr beliebt.
Beispielsweise wurde Kubernetes in dieser Sprache entwickelt

\section{Sprachkonstrukte}
\begin{wrapfigure}[21]{r}{5.5cm}
	\includegraphics[width=5.5cm]{fibGoroutine}
	\caption{Beispiel Goroutine, Channel und Select}\label{img: fibGoroutine}
\end{wrapfigure}
\subsection{Goroutines, Channels \& Select}

\paragraph{Goroutines}
Eine Goroutine ist ein leichtgewichtiger Thread, welcher von der Go runtime gemanaged wird.
Goroutines nutzen dabei den selben Adressraum, daher muss der Zugriff auf geteilte Ressourcen synchronisiert werden.
In Go gibt es Primitives, welche die Synchronisierung übernehmen. Jedoch werden diese nur selten genutzt, da in den meisten Fällen mit Channels gearbeitet wird.
\paragraph{Channels}
Ein Channel erlaubt einen einfachen Datenfluss. Per Default ist dabei das Senden und Empfangen blockierend, bis die andere Seite bereit ist. Goroutines lassen sich somit sehr leicht synchronisieren, ohne den Einsatz von Locks oder Variablen. Der Channel kann explizit vom Sender geschlossen werden. Damit wird dem Empfänger signalisiert, dass keine Werte mehr empfangen werden können. Das Schliessen des Channels sollte dabei exklusiv vom Sender ausgeführt werden. Senden auf einen geschlossenen Channel verursacht dabei "panic".


\paragraph{Select}
Bei einem Select wird bei mehrfacher Auswahl gewartet, bis eine Operation laufen kann.

\subsection{Maps \& Slices}
\begin{wrapfigure}[14]{r}{7cm}
	\centering
	\includegraphics[width=7cm]{slices}
	\caption{Verhalten von Slices, Quelle Autor}\label{slices}
\end{wrapfigure}
\paragraph{Slices}

Bei Slices handelt es sich um ein dynamisches Array welches speziell für Go implementiert wurde und als Datenstruktur in der Standardbibliothek mitenthanlten ist.
Im Untergrund besteht ein Slice aus einem Array weshalb es auch Funktionen eines Arrays mitliefert.
Ein wichtiger Unterschied ist, dass ein Array ein Value-Typ ist, wohingegen ein Slice ein Referenz-Typ.

Bei der Instanzierung eines Slices kann dessen Länge sowie Kapazität angegeben werden.
Die Länge bezeichnet die Anzahl Elemente welches ein Slice enthält und die Kapazität die Anzahl Elemente welche sich im unterliegenden Array befinden.

\paragraph{Maps}
\begin{wrapfigure}[12]{l}{6cm}
	\centering
	\includegraphics[width=6cm]{map-example}
	\caption{Verhalten von Slices, Quelle Autor}\label{map}
\end{wrapfigure}
Hierbei handelt es sich um eine Key-Value-Datenstruktur, welche in anderen Sprache unter HashMaps bekannt ist.
Typen von Keys müssen comparable sein, weshalb Funktionen und Slices nicht genommen werden können.
Eine spezielle Eigenschaft dieser Map ist der Value-Type \textbf{Interface}.
Damit steht als Value dem Entwickler offen was für Typen dieser dort nutzen will.
Es kann eine Mischung aus Structs, Strings, Arrays, Pointers etc. sein.
Diese Eigenschaft ist nützlich, wenn bspw. das Schema vom erhaltenen JSON nicht bekannt ist und in einer Map gespeichert werden soll.
map interface!

\begin{wrapfigure}[15]{r}{5.5cm}
	\centering
	\includegraphics[width=5.5cm]{structuralTyping}
	\caption{Structural Typing in Go, \\Quelle Autor}\label{structuralTyping}
\end{wrapfigure}
\subsection{Structural \& Nominal Typing}
Beim nominal Typing wird gegen den Namen verglichen, wohingegen beim structural Typing die Struktur beachtet wird. Objektorientierte Sprachen wie Java, C++ oder auch Swift tendieren zum nominal Typing, wohingegen funktionale Sprachen eher zu structural Typing tendieren. \\
Bei dynamisch typisierten Sprachen existiert zusätzlich DuckTyping. Diese verfolgen den Ansatz: It's easier to ask forgiveness than permission. Bei go gibt es kein DuckTyping. Das ist auch im Beispiel zu sehen. Es wird nicht versucht, den Code laufen zu lassen, sondern direkt ein Compilerfehler angezeigt.
Go erlaubt es daher, einem Objekt vom Typ Meter eine Instanz vom Typ Centimeter zuzuweisen. In Sprachen mit einem Nominal Typing wäre dies nicht möglich. Damit dies möglich ist, muss die Struktur der beiden structs identisch sein. Go unterscheidet auch zwischen den vordefinierten Typen wie int64 und int32. 


\subsection{The Go Memory Model}
\begin{wrapfigure}{l}{0.4\linewidth}
	\centering
	\includegraphics[width=6cm]{memorymodel}
	\caption{Beispiel vom Memory Management, Quelle Autor}\label{memorymodel}
\end{wrapfigure}
Die Sprache \textbf{Go} liefert ein automatisches Memory Management mit wie zum Beispiel automatische Memoryallozierung oder den Garbage Collector.
Dieser arbeitet dabei mit Memory Blocks, welche Values beinhalten und unterschiedlich gross sein können.
Memory Blocks entstehen bspw. durch den Aufruf von \textbf{new} (nur einen auf dem Heap oder Stack) oder \textbf{make} (mehrere auf dem Heap), Variablendeklarationen etc.
Jede goroutine verwaltet einen Stack (Memory Segment).
Dabei handelt es sich um einen Memory Pool von Memory Blocks welche alloziert werden können.
Memory Blocks, welche durch eine Goroutine auf einen Stack alloziert werden, sind nur in dieser Goroutine intern ersichtlich.
Darin müssen keine Synchronisationsarbeiten vorgenommen werden.
Beim Heap handelt es sich um ein Singleton.

Wenn ein Memory Block nicht in einer Goroutine Stack alloziert ist, landet er im Heap.
Solche Memory Blöcke können durch mehrere Goroutine verwendet werden, weshalb die Synchronisierung beachtet werden muss.
Wenn der Compiler entdeckt, dass ein Memory Block über mehrere Goroutines hinweg verwendet wird oder sich nicht sicher ist ob es auf den Stack soll, wird dieser im Heap abgelegt.
Die Vorteile Memory Blocks auf dem Stack abzulegen sind, dass die Allozierung schneller ist, diese nicht durch den Garbage Collector bereinigt werden müssen und sie CPU Cache-freundlicher sind.

\subsection{Package Management}
In 2018 wurde das Go-Modules Konzept eingeführt. Go Modules können intern im Projekt, jedoch auch von externen Projekten importiert werden. Ein weiterer Grund für Go Modules ist die Vereinfachung des Dependency Managements. Im Module File werden dabei alle Dependencies gesammelt und ermöglichen reproduzierbare Builds. Es ist auch möglich, verschiedene Versionen eines Modules als Dependency hinzuzufügen. In einem anderen File speichert go die Hashes der Dependencies. Dadurch wird sicher gestellt, dass die Versionen nicht verändert wurden. Die bereits heruntergeladenen Module werden gecached und im Projektpfad abgelegt. \\
Im Beispiel wird auf ein lokales Modul zugegriffen. Hierzu wird ein replace des Pfads gemacht, da das Modul nicht aus dem Internet herunterladbar ist. Es wird auf das Dateisystem verwiesen. Ausserdem ist die Versionsnummer zufällig gewählt. 

\subsection{Defer}


\section{Fazit}
\subsection{Team-Fazit}
Der Einstieg in die Go-Welt war dank dem Tutorial \glqq Tour of Go\grqq{} nicht besonders anspruchsvoll.
Dabei wurden die wesentlichen Sprachkonstrukte anhand von einfachen, nachvollziehbaren Beispielen eingeführt.
Es wurde genug Dokumentation zu dieser Sprache gefunden und es stellte sich heraus, dass diese eine grosse Community aufweist.
Da die Sprache ebenfalls an C angelehnt ist, ist die Syntax als Java-Entwickler nicht befremdent.
Das Lösen der Aufgaben bereitete nur bedingt Probleme.
Obwohl es sich bei Go um eine Multiparadigmensprache handelt, wurden die Übungen meistens mit objektorientierten Ansätzen gelöst.
Es wurden aber mit Absicht Übungen selektiert, welche andere Paradigmen verfolgen als die Objektorientierung. \\
Resultierend aus den gesammelten Erfahrungen fällt das Teamfazit sehr positiv aus.
Beide Teammitglieder würden die Sprache in grösseren Projekten gerne mal einsetzen.

\subsection{Fazit Frederico Fischer}
Als DevOps-interessierte Person bin ich beruflich als auch privat bei diversen Projekten auf diese Sprache angestossen, weshalb ich diese anhand dieses Projektes genauer anschauen wollte.
Mir gefällt besonders, dass man sehr schnell kleinere Scripts oder Tools damit entwickeln kann, jedoch auch mit grossen Projekten zurecht kommt.
Ich finde es sehr interessant, wieviel Wert auf das Memory Management, die Kompilierung sowie die Nebenläufigkeitsansätze gesetzt wurde.
Was ich jedoch vermisst habe sind gewisse Datenstrukturen.
Zwar sticht es mit Slices oder Maps mit Interfaces als Value-Types heraus, besitzt aber standardmässig keine weiteren bekannten Datenstrukturen wie Sets oder Queues.
Ich würde mich aber trotzdem gerne beruflich weiter mit dieser Sprache auseinandersetzen.

\subsection{Fazit Oliver Werlen}
Go bietet viele, von anderen Sprachen bekannte, Konstrukte direkt von Haus aus und vereinfacht deren Anwendung.
Besonders das Schreiben von asynchronem Code fiel mir positiv auf, da die Verwendung von Channels die Implementierung solcher Synchronisierungsmechanismen sehr vereinfacht.
Das Modul Konzept von Go macht den Einsatz eines zusätzlichen Dependency Managers wie Gradle oder Maven überflüssig. \\
Go müsste sich bei mir zuerst in einem grösseren Projekt zum Einsatz kommen, um ein abschliessendes Fazit geben zu können.
Der erste Eindruck ist aber positiv.

    

\end{document}
